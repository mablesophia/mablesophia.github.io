\chapter{Introduction to Time Series Analysis}
A {\sl time series} is a collection of observations made
sequentially in time. Examples are daily mortality counts,
particulate air pollution measurements, and temperature data.
Figure 1 shows these for the city of Chicago from 1987 to 1994.
The public health question is whether daily mortality is
associated with particle levels, controlling for temperature.

\centerline{\epsfig{figure=Plots/plot-10-01.ps,angle=270,width=\textwidth}}

We represent time series measurements with $Y_1, \dots, Y_T$
where $T$ is the total number of measurements. In order to analyze
a time series, it is useful to set down a statistical model in the
form of a {\sl stochastic process}. A stochastic process can be
described as a statistical phenomenon
that evolves in time. While most statistical problems are
concerned with estimating properties of a population from a
sample, in time series analysis there is a different situation.
Although it might be possible to vary the length of the observed
sample, it is usually impossible to make multiple observations at
any single time (for example, one can't observe today's mortality
count more than once). This makes the conventional statistical
procedures, based on large sample estimates, inappropriate.
Stationarity is a convenient assumption that permits us to
describe the statistical properties of a time series.


\section{Stationarity}
Broadly speaking, a time series is said to be {\sl stationary} if
there is no systematic trend, no systematic change in variance,
and if strictly
periodic variations or seasonality do not exist. Most processes in
nature appear to be non-stationary. Yet much of the theory in
time-series literature is only applicable to stationary processes.

One way of describing a stochastic process is to specify the joint
distribution of the observations $Y(t_1), \dots, Y(t_n)$ for any
set of times $t_1, \dots, t_n$ and any value of $n$. A time series
is said to be strictly stationary if the joint distribution of
$Y(t_1), \dots, Y(t_n)$ is the same as that of $Y(t_1 + h), \dots
Y(t_n + h)$ for all $t_1, \dots, t_n$ and $h$. To see how this is
a useful assumption, notice that the above condition implies that
the expected value and covariance structure of any two components,
$Y_a(t)$ and $Y_b(t)$, of a time series are
constant in time
\begin{equation}
\label{sos}
\E \{Y_a(t)\} = \mu_a \, , \, \var \{Y_a(t)\} = \sigma_a^2 \mbox{ and }
\mbox{corr}\{ Y_a(t), Y_b(t+h) \} = \gamma_{ab}(h).
\end{equation}
The function
$\gamma_{ab}(h)$ is called the cross-correlation function if $a\neq b$
and the auto-correlation function if $a=b$. 

In practice it is often useful to define stationarity in a less
restricted way than that described above. In many cases,
the statistical structure of the processes can be
completely described with the second-order properties of equation
(\ref{sos}). We can estimate the
quantities in (\ref{sos}) using
standard statistical procedures,
for example we may estimate the cross-correlation at {\it lag}
$h$, $\gamma_{a,b}(h)$ with the sample correlation of
$Y_a(1),\dots,Y_a(T-h)$ and $Y_b(h+1),\dots,Y_b(T)$. 



\subsection{An example: Fetal Monitoring}

Measurements of fetal heart rate (FHR) and fetal movement (FM) are
generated by maternal-fetal monitoring.    
Approximately 5 measurements per second are taken during 50 minutes on
120 subjects that are monitored at 20,24,28,32,36,
and 38-39 weeks of 
gestation. Both FHR and FM are recorded giving us a multiple time 
series $Y(t), t=1,\dots,50\times 60\times 5$, where $Y(t)$ is a vector
with 2 entries. 

The association between accelerations of FHR and    
FM has been documented since the 1930s. For
example, it has been observed 
that in the third trimester most large fetal heart
accelerations are 
associated with fetal activity.
In Section 4.2 we will describe how relatively straight-forward time
series techniques provide a visual descriptions of how these
associations vary with weeks gestation. These description have
motivated a methodology that will provide us is with a more rigorous
assessment of this relationship.

If we consider the FM and FHR  measurements, seen in Figure 5, as
outcomes from a two component time series, we may consider the
cross-correlation function of these two components as a description of
the association between these two processes. Notice that the
measurements taken for each 
fetus at each gestation week has a cross-correlation function associated
with them. In Figure 6, as a descriptive plot, we show the average,
over individuals,
of these functions for each gestation
week. Notice that a peak at around the $-6$ second lag starts to
appear in the plot for the 24 week gestation. As the fetus gets older,
this peak grows and becomes more defined. This result
can be 
considered a first step in the characterization of the relationship
between FM and FHR. 

\centerline{\epsfig{figure=Plots/plot-10-02.ps,width=\textwidth}}
\centerline{\epsfig{figure=Plots/plot-10-03.ps,width=\textwidth}}

\section{Spectral Analysis}

Sometimes it is useful to describe the properties of the time series
in a frequency domain. The spectrum is defined as 
\[
f_{ab}(\lambda) = \frac{\sigma^2}{2\pi} \sum_{h = -\infty}^{\infty} \gamma_{ab}(h) \exp
( -i \lambda h)
\]

There is a one-to-one correspondence between the spectrum and the
autocovariance function 
\[
\sigma^2 \gamma_{ab}(h) = \int_{-\pi}^{\pi} f(\lambda) \exp ( i\lambda h) d\lambda
\]

We call $|f_{aa}|^2$ the power spectrum. A natural way of estimating a
power 
spectrum is using the periodogram which 
is the modulus of the Fourier transform of the data
\[
I(\lambda) = \frac{1}{2\pi T} | \sum_{t=1}^T Y_t \exp (-i \lambda t) |^2
\]
We usually compute the periodogram at the Fourier frequencies
$\lambda_j = (2 \pi j)/T, j=1,\dots,T/2$. These have desirable
statistical properties

The periodogram is also useful for detecting periodicities
(deterministic ones) in the signal. It is a mathematical fact that if
the data $Y_1,...Y_T$ has a period $p$, the the periodogram will have
peaks at frequencies $\lambda = 2 \pi T/p$ and its multiples. 

\centerline{\epsfig{figure=Plots/plot-10-04.ps,angle=270,width=\textwidth}}


\subsection{An Application}
Figured 4a and 4c shows recorded ECoG signal for two channels for
a subject that has received a sensory stimulus at some point
during the recording. A straightforward way of estimating the
spectrum of a stationary process is the periodogram
\[
I(\lambda) = \frac{1}{2\pi T} \left| \sum_t Y(t) \exp(i\lambda t)
\right|^2.\]
In Figures 4b and 4c the periodogram of this data is shown. Brain
researchers have speculated that the so-called $\alpha$ $(8-13
Hz.)$, $\beta$ $(15-25 Hz.)$, and $\gamma$ $(>30 Hz.)$ bands of
human brain signals can indicate functional activation of
sensorimotor cortex. Notice that the
periodogram exhibits a peak around frequencies 10 Hz., 20 Hz. and
60 Hz.. If we were to approximate the ECoG signal as a stationary
processes, we would describe it as having periodic components
around these frequencies. However, we are interested in learning
how the signal changes when the subjects are given a stimuli. Thus
it seems more appropriate to model the signal as a non-stationary
processes and study the time-varying spectral density.

A straightforward estimate of a time-varying spectral density
would be the dynamic periodogram. Basically, for each time $t_0$
we consider a window around that point of size $h(t_0)$ and
estimate a weighted periodogram
\[
I(t_0;\lambda) = \frac{1}{2\pi h(t_0)} \left| \sum_t
w\left(\frac{t-t_0}{h(t_0)}\right) Y(t) \exp(i\lambda t) \right|^2.
\]
Figures 4c and 4e show the estimated time-varying spectral
densities for the signals of channels 19 and 20 (lighter colors
represent higher values)
The figure seems to suggest that the $\alpha$ band changes power
and frequency after the stimulus (time 0).


\centerline{\epsfig{figure=Plots/plot-10-05.ps,width=\textwidth}}
\centerline{\epsfig{figure=Plots/plot-10-06.ps,width=\textwidth}}
\centerline{\epsfig{figure=Plots/plot-10-07.ps,width=\textwidth}}
