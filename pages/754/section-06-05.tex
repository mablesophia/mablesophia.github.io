\section{Bayesian Model for Cubic Splines}
Notice that the above definition can be easily extended to a Bayesian
problem: If we define a distribution for $f$, say $dP(f)$ then we may
consider the average risk
\[
\int_f  \E [ || \hat{g} - f ||^2 | f] \, dP(f)
\]
as a criterion. 

This follows section 3.6 in the book by Hastie and Tibshirani.

The cubic smoothing spline can be derived from a number of Bayesian
models for smoothing. The details are hard, but can be found in a book
by Wahba.

Here we will demonstrate a fairly simple example.

Remember we can write any natural cubic spline as 
\[
g(x) = \bB(x) \bg{\theta}.
\]
Consider the following Bayesian set-up:

Model assumptions: Assume the data $\by$ follow a Gaussian distribution $N(\bB
\bg{\theta}, \sigma^2 {\mathbf I}_n)$. 

Prior assumptions: Assume $\bg{\theta}$ follows a multivariate
Gaussian prior distribution with mean $0$ and variance
$\sigma^2/\lambda  \bg{\Omega}^{-1}$. 

it follows that the posterior distribution of $\bg{\theta}$ is
multivariate Gaussian with mean 
\[
\E(\bg{\theta}|\by) = \bB(\bB'\bB +  \lambda \bg{\Omega})^{-1} \bB'\by
\]
which is the natural smoothing spline estimate of $\bg{\theta}$.







