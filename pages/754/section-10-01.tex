\section{Stationarity}
Broadly speaking, a time series is said to be {\sl stationary} if
there is no systematic trend, no systematic change in variance,
and if strictly
periodic variations or seasonality do not exist. Most processes in
nature appear to be non-stationary. Yet much of the theory in
time-series literature is only applicable to stationary processes.

One way of describing a stochastic process is to specify the joint
distribution of the observations $Y(t_1), \dots, Y(t_n)$ for any
set of times $t_1, \dots, t_n$ and any value of $n$. A time series
is said to be strictly stationary if the joint distribution of
$Y(t_1), \dots, Y(t_n)$ is the same as that of $Y(t_1 + h), \dots
Y(t_n + h)$ for all $t_1, \dots, t_n$ and $h$. To see how this is
a useful assumption, notice that the above condition implies that
the expected value and covariance structure of any two components,
$Y_a(t)$ and $Y_b(t)$, of a time series are
constant in time
\begin{equation}
\label{sos}
\E \{Y_a(t)\} = \mu_a \, , \, \var \{Y_a(t)\} = \sigma_a^2 \mbox{ and }
\mbox{corr}\{ Y_a(t), Y_b(t+h) \} = \gamma_{ab}(h).
\end{equation}
The function
$\gamma_{ab}(h)$ is called the cross-correlation function if $a\neq b$
and the auto-correlation function if $a=b$. 

In practice it is often useful to define stationarity in a less
restricted way than that described above. In many cases,
the statistical structure of the processes can be
completely described with the second-order properties of equation
(\ref{sos}). We can estimate the
quantities in (\ref{sos}) using
standard statistical procedures,
for example we may estimate the cross-correlation at {\it lag}
$h$, $\gamma_{a,b}(h)$ with the sample correlation of
$Y_a(1),\dots,Y_a(T-h)$ and $Y_b(h+1),\dots,Y_b(T)$. 



\subsection{An example: Fetal Monitoring}

Measurements of fetal heart rate (FHR) and fetal movement (FM) are
generated by maternal-fetal monitoring.    
Approximately 5 measurements per second are taken during 50 minutes on
120 subjects that are monitored at 20,24,28,32,36,
and 38-39 weeks of 
gestation. Both FHR and FM are recorded giving us a multiple time 
series $Y(t), t=1,\dots,50\times 60\times 5$, where $Y(t)$ is a vector
with 2 entries. 

The association between accelerations of FHR and    
FM has been documented since the 1930s. For
example, it has been observed 
that in the third trimester most large fetal heart
accelerations are 
associated with fetal activity.
In Section 4.2 we will describe how relatively straight-forward time
series techniques provide a visual descriptions of how these
associations vary with weeks gestation. These description have
motivated a methodology that will provide us is with a more rigorous
assessment of this relationship.

If we consider the FM and FHR  measurements, seen in Figure 5, as
outcomes from a two component time series, we may consider the
cross-correlation function of these two components as a description of
the association between these two processes. Notice that the
measurements taken for each 
fetus at each gestation week has a cross-correlation function associated
with them. In Figure 6, as a descriptive plot, we show the average,
over individuals,
of these functions for each gestation
week. Notice that a peak at around the $-6$ second lag starts to
appear in the plot for the 24 week gestation. As the fetus gets older,
this peak grows and becomes more defined. This result
can be 
considered a first step in the characterization of the relationship
between FM and FHR. 

\centerline{\epsfig{figure=Plots/plot-10-02.ps,width=\textwidth}}
\centerline{\epsfig{figure=Plots/plot-10-03.ps,width=\textwidth}}
